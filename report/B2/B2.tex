\documentclass[UTF-8]{ctexart}
\usepackage{graphicx}
\usepackage{pdfpages}
\usepackage{minted}
\title{基于B2的探测器模拟}
\begin{document}
\maketitle
\section{代码思路}
在 \texttt{B2aDetectorConstruction.cc}中修改探测器几何与\texttt{ConstructSDandField()}函数。

探测器几何修改包括对其中的探测器物质和几何的修改。

在 \texttt{B2PrimaryGeneratorAction.cc}中设置粒子能量和粒子出射方向,位置。

在新创建的\texttt{B2a.mac}中加入模拟时的代码,对每种粒子模拟10次

\begin{minted}{sh}
/globalField/setValue 0 0 0.5 tesla
/gun/energy 1 GeV

/gun/particle mu+
/run/beamOn 10
/vis/ogl/export mu.eps

/gun/particle proton
/run/beamOn 10
/vis/ogl/export proton.eps

/gun/particle e-
/run/beamOn 10
/vis/ogl/export e.eps

/gun/particle neutron
/run/beamOn 10
/vis/ogl/export neutron.eps
\end{minted}
\section{模拟结果}
\subsection{mu+}
$\mu^+$,可以看到几乎没有产生反应

    \includegraphics[width=1.1\textwidth]{../../B2/B2a-build/mu_0000.eps}
\subsection{e-}
$e^-$,在闪烁体中产生大量次级粒子,并且向四周发射,注意正中央是粒子出发点

\includegraphics[width=1.1\textwidth]{../../B2/B2a-build/e_0000.eps}
\subsection{proton}
$p$,和$e^-$相当类似

\includegraphics[width=1.1\textwidth]{../../B2/B2a-build/proton_0000.eps}
\subsection{neutron}
$n$即便在空气中也没有轨迹

    \includegraphics[width=1.1\textwidth]{../../B2/B2a-build/neutron_0000.eps}
\end{document}